\documentclass{article}
% kodowanie: latin2, utf8 lub cp1250
\usepackage[polish]{babel}
\usepackage[utf8]{inputenc}
\usepackage{polski}
\usepackage[T1]{fontenc}
\frenchspacing
\begin{document}
Złoty podział wykorzystuje się często w estetycznych, proporcjonalnych kompozycjach architektonicznych, malarskich, fotograficznych itp. Znany był już w starożytności i przypisywano mu wyjątkowe walory estetyczne. Stosowano go np. w planach budowli na Akropolu. Co najmniej od XX wieku wielu artystów i architektów tworzyło swoje dzieła z zachowaniem złotego stosunku - szczególnie w formie złotego prostokąta, w którym stosunek dłuższego boku do krótszego jest równy złotej proporcji - zgodnie z poglądem, że takie proporcje wyglądają estetycznie. Złoty prostokąt może być rozcięty na kwadrat i mniejszy prostokąt o tych samych proporcjach co rozcinany. Matematycy, począwszy od Euklidesa, badali złoty podział z powodu jego wyjątkowych i interesujących własności. Złoty podział jest także używany w analizie rynków finansowych, w strategiach takich jak odbicie Fibonacciego.
\\\\\\
 Dwie wielkości a i b są w złotym stosunku $\Phi$, jeżeli:
\begin{equation}
\frac{a+b}{a} = \frac{a}{b} = \Phi
\end{equation}
 Równanie \begin{math} \Phi^{2} = 1 + \Phi \end{math} również produkuje łańcuchowy pierwiastek kwadratowy, to znaczy:
\begin{equation}
\Phi = \sqrt{ 1+\sqrt{1+{\sqrt{1+{\sqrt{\ldots}}}}}}
\end{equation}

\end{document}
